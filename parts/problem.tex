\kapitola{Definice problému}
\sekce{HTML šablona}
Webová šablona je univerzální vzhled webové stránky. Je vytvořená programátory a~designéry. Důležitý parametr dobré šablony je, aby s~minimální modifikací, vyhovovala co největšímu počtu uživatelů. Šablona musí být validní, dle posledních W3C standardů \cite{cameron}. 

Šablony jsou programovány pomocí značkovacího jazyka HTML, je nutné pro jejich úpravu tento jazyk ovládat. Většina běžných počítačových uživatelů tento jazyk ovšem neovládá. V~případě, že by nalezli na internetu vyhovující šablonu jejich účelu, nebudou schopni ji bez pomoci odborníků upravit pro svoje potřeby. 

\sekce{Publikování}
K~procesu vytváření a~editování webové stránky jednoznačně patří i~její zveřejnění na internetu \cite{zandstra2010php}. Pro zveřejňování webových stránek se ve většině případů používá sdílený hosting. K~zakoupenému hostingu uživatel dostane údaje k~přístupu na úložiště přes FTP (File Transfer Protocol). Na základě těchto údajů je nutné připojit se k~úložišti a~přesunout tam webové stránky. Dále je nutné zakoupit a~nakonfigurovat doménu. I~přes poměrně složitý postup ještě není dosaženo optimálních výsledků. Je nutné splnit ještě několik kroků pro zabezpečení webových stránek. 

Tato činnost může méně zkušeného uživatele odradit a~raději tento problém řeší obrácením se na odbornou firmu. Proto je nutno i~tuto činnost zjednodušit lidem, kteří si chtějí vytvořit, nebo editovat webové stránky a~nemají potřebné znalosti. Mezi nejjednodušší řešení tedy patří, když uživatel nebude nucen mít vlastní hosting, který je nutný konfigurovat, přesouvat tam soubory a~podobně, ale naopak, když bude všechno obsluhovat jeden společný server redakčního systému, o~který se nebude muset uživatel starat.

\sekce{Současné možnosti}
Osoba bez odborných počítačových znalostí, která se chce prezentovat na internetu webovými stránkami, má v~současné době jen několik možností. Jedna z~nich je vytvořit webové stránky pomocí redakčních systému, jako je Wordpress nebo Joomla. Tato možnost je omezující z~důvodu nutnosti využít šablony, které jsou přímo naprogramovány pro tento redakční systém. Takto upravených šablon není na internetu dostatek. I~z~toho důvodu se nezřídka stává, že na internetu lze nalézt stejné webové stránky, jen s~odlišným textem.  Další možností, finančně velmi nákladnou, je vytvořit poptávku u~profesionální firmy, zabývající se tvorbou webových stránek na míru \cite{arlow}.