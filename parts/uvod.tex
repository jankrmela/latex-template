\kapitola{Úvod here}
\sekce{Úvod do problematiky}
Na internetu je k~dispozici tisíce šablon, napsaných v~jazyce HTML, které není možné upravit k~vlastní potřebě, průběžně je aktualizovat a~publikovat na internetu, bez odborných počítačových znalostí. Ve většině případů, jsou tyto HTML šablony propracované a~mohou sloužit jako elegantní webová prezentace \cite{krug}. Bohužel v~současné době není řešení, jak tyto šablony mohou lidé se základními počítačovými znalostmi využít. Uživatelé tedy musí vybírat z~omezeného množství webových šablon, naprogramovaných pro konkrétní redakční systém \cite{cameron}.

HTML šablona obsahuje obrázky, CSS soubory, Javascript soubory a~zdrojové kódy napsané v~jazyce HTML. Všechny tyto části dohromady tvoří webovou stránku, kterou zobrazují webové prohlížeče. Editace webové stránky je možná změnou zdrojového kódu. Tato činnost ovšem vyžaduje odborné znalosti. Z~toho důvodu běžní uživatelé využívají redakční systémy, ve kterých zvládne úpravu webové stránky každý, kdo má základní zkušenosti s~používáním počítače. Webová stránka se pak edituje podobně jako dokumenty v~textovém editoru. 

Úprava webové stránky v~redakčním systému je tedy velmi pohodlnou záležitostí. Jádro problému je ovšem v~obtížnosti implementace HTML šablony do redakčního systému. Každou HTML šablonu je nutné speciálně upravit, aby redakční systém rozuměl dané šabloně a~dokázal nabídnout uživateli komfortní editaci. Programátor tedy musí použít další programovací jazyk, nejčastěji PHP, kterým šablonu upraví, rozdělí ji do samostaných částí, vytvoří databázi a~provede další speciální kroky pro konkrétní redakční systém. Tento proces je bohužel povinný a~zdlouhavý. V~současné době neexistuje řešení, jak bez této úpravy editovat HTML šablony pohodlně z~prohlížeče, bez znalosti programovacích jazyků.

\sekce{Cíl práce}
Navrhované řešení je vytvořit webovou aplikaci, která by byla schopna zpracovat libovolnou validní HTML šablonu, nabídnout uživateli pohodlnou možnost editace z~prohlížeče a~zveřejnění takové webové prezentace. Důležitou podmínkou je, aby tuto činnost zvládla i~osoba bez pokročilých počitačových znalostí. 

V~současné době dokáží upravit a~publikovat HTML šablonu pouze uživatelé se znalostí programování. Navrhované řešení by umožnilo používání všech HTML šablon i~pro méně zdatné počítačové uživatele. 

