\kapitola{Úvod do problematiky}
Společnost Kiwi.com je původem česká online služba s celosvětovým dosahem. Obrat firmy byl v roce 2018 tvořil 31 miliard korun. \cite{stratilik_trzby_nodate}
Kiwi.com zprostředkovává uživatelům nákup letenek různých leteckých společností. Hlavní funkcionalita služby spočívá v umožnění najít spoj z libovolného místa A do místa B za velmi výhodnou cenu. Unikátní vyhledávací algoritmus dokáže poskytnout velmi optimální výsledky, diky propojení stovek leteckých společností i pozemní dopravy. Denně přes portál Kiwi.com zákaznící obsadí 35 tisíc sedadel. \cite{dlouhy_kiwi.com_nodate}

Společnost byla označena magazínem Forbes za startup roku 2018. Kiwi.com také obdrželo mezinárodní ocenění za řešení globalních platebních transakcí v rámci TMI Awards. Denně Kiwi zprostředkovává 100 milionu vyhledávání. \cite{mares_startup_2018}% || Forbes

Obchodním modelem je ale unikátní princip. Po nalezení vhodného itineráře, který může obsahovat spojení přes několik dopravních společností uživatel vyplní pouze jednu objednávku na webu Kiwi.com a společnost vystaví vlastní lístek pro tento let. Zákazník tedy nemusí nakupovat jednotlivě na webech jednotlivých společností. 

Kiwi.com zároveň nabízí záruku na takto zakoupené lístky. Slibuje, že v případě zpoždění jednoho z letů, poskytne náhradu za navazující spoje letu a v některých případech uhradí i ubytování v daném městě, pokud by zákazník musel na náhradní let čekat. FIrma pak náhradu, kterou poskytuje přes takzvanou Kiwi.com Guarantee, musí získat zpět od leteckých společností. Vymáhání náhrady od leteckých společností, ale není snadné, protože každá letecká společnost má své pravidla, kdy tyto náhrady proplácí. 

Firma Kiwi.com musí tedy rozvážně zvažovat, zda zákazník má právo na poskytnutí náhrady a zda je možné získat peníze zpět od dopravní společnosti, pokud by tato náhrada byla zákazníkovi vyplacena. Tuto odpovědnost mají jednotliví zaměstnanci zákaznické podpory. O každém případu se rozhoduje individuálně. Tento proces není ideální, protože může často docházet k lidské chybě a celkově je tento proces časově náročný a nelze automatizovat. Rovněž může nastat nespokojenost na straně zákazníka. Zákazník totiž nemusí přesně vědět, co se skrývá pod pojmem Kiwi Guarantee a neví v jakých případech dostane odškodnění. Na internetu existuje mnoho recenzí od zákazníků, kteří očekávali poskytnutí náhrady firmou Kiwi.com, ale neobdrželi ji, protože jim na zákaznické podpoře bylo sděleno, že na poskytnutí náhrady nemají nárok. 

Naopak mnoho zákazníků má s Kiwi Guarantee i pozitivní zkušenosti, protože jim Kiwi nabídlo a zarezervovalo náhradní let, včetně ubytování, pokud by museli čekat na přestupním letišti více dní. Celkově z těchto informací vyplývá problém o nízké informovanosti uživatele při procesu objednávky. Není tedy pro uživatele zřejmé, které případy pokrývá Kiwi záruka a které případy ne. Zároveň pro Kiwi.com je problém ve finanční náročnosti poskytovat tuto garanci všem a zadarmo. 


\kapitola{Cíl práce}
Cílem diplomové práce je implementovat na frontendové části webové služby Kiwi.com novou nabídku tarifů a doplňkových balíčků, které obsahují různou míru doplňkových služeb, pojištění, zákaznickou podporu a dalších nadstandartních služeb. 

Od zavedení tarifů a balíčků si firma slibuje vyšší zisk, kterým lze pokrýt náklady spojené s poskytováním náhrad zákazníkům. Cílem je tedy nabídnout tarify a balíčky tak, aby se maximalizovat prodej těchto balíčků, bez negativního vlivu na celkovou konverzi v objednávkovém procesu.
Dalším požadavkem na novou nabídku tarifů je informovat uživatele již v procesu objednávky jaké nadstandardní služby a záruky jsou v ceně a motivovat zákazníky k zakoupení placených tarifů a balíčků, které by zmírnili negativní reakce v případě problémů. 

Součástí řešení je také problém nejednotného měřítka při poskytování náhrad zákazníkům, které v současné době je řešeno individuálně zaměstnanci s každým zákazníkem. Vhodné řešení by mělo určit exaktní měřítko při poskytování náhrad a vyplácení pojištění. Řešení by mělo zároveň umožnit automatické zpracování části požadavků, na základě kterého by bylo možné ušetřit náklady na zaměstnávání lidské pracovní síly, která musí požadavky zpracovávat ručně. 

Implementace nové nabídky nesmí negativně ovlivnit objednávkový proces, tak aby neklesla celková hodnota objednávek provedená na portálu Kiwi.com. Součástí práce musí být rozhodnutí o ideálním zasazením nové nabídky tarifů a balíčků do správného místa v procesu objednávky, stejně tak jako zvolení správného vizuálního rozložení prvků. 

V této práci je nutné pro dosažení optimálních výsledků provést důkladnou analýzu požadavků, včetně zajištění, aby se nové řešení a jeho dopady shodovaly s vizí firmy. Před technickou implementací bude zásadní analyzovat a zhodnotit konkurenční prostředí, které se dotýká trhu, na kterém působí i Kiwi.com. Při analýza současného řešení se definují vstupní a výstupní podmínky, činnosti procesu a podniková pravidla, dle kterých pak bude následně sestaven návrh řešení, včetně funkčních a nefunkčních požadavků na systém. Návrh systému bude vytvořen pomocí jazyka UML. Následná implementaci nové nabídky musí být integrována do současného řešení a napojena na stávající systémy. 