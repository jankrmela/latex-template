\kapitola{Ekonomické přínosy}

Primárním požadavkem firmy bylo celkové zvýšení zisku z prodeje letenek. Nelze ovšem označit hodnotu prodaného tarifu a balíčku jako stoprocentní zisk. Se zakoupeným tarifem a balíčkem jsou spojeny i náklady pro firmu. Do nákladů se musí započítat i zvýšené náklady spojené s poskytováním zákaznické podpory, poskytování náhrad zákazníkům, na které mají v rámci zaplacených tarifů nárok, náklady na poskytování podpory emailem a podobně. 

Společnost Kiwi.com trvá na utajení hodnoty těchto nákladů vzhledem k hodnotě tarifů a tedy i celkového zisku. Stejně tak není možné zveřejnit zisk, průměrnou cenu objednávek a ani jiné data, které by mohly ohrozit postavení firmy v silně konkurenčním prostředí. Není tedy možné v této práci uvést zisk z prodeje balíčků. Úspěšnost návrhu a implementace bude vyhodnocen pomocí procentuálního poměru prodeje placených balíčků. Placené tarify a balíčky totiž jednoznačně přináší firmě zisk. Cena těchto balíčků byla zvolena firmou výpočtem, který zajistil celkovou ziskovost spojenou s prodejem balíčků.

Základní tarif Saver a balíček Basic je zdarma a neplynou z něj ani náklady navíc. Na verzi nákupu, kde si zákazník vybral tyto služby zdarma, lze nahlížet stejně, jako na původní verzi, před implementací nového řešení. A jakýkoliv nákup placeného tarifu a placeného balíčků, lze považovat jako zisk navíc, způsoben právě touto implementací nové nabídky. 

Posouzení úspěšnosti navrhovaného řešení spočívá v určení procentuálního množství objednávek, kde je vybrán některý z placených tarifů a balíčků. Celkové vyhodnocení pak bude probíhat na základě dopadu těchto hodnot na hypoteticky zvolené data, nebo data, které již firma dříve zveřejnila sama.

Nově implementovaná nabídka byla z důvodu opatrnosti v rámci AB testu zobrazována pouze 50 \% návštěvníkům. Z dat bylo ale okamžitě po prvním dni zřejmé, že nová funkcionalita má velmi pozitivní výsledky. Proto se AB test deaktivoval a zvolila varianta se zobrazováním všem zákazníkům, jako vítězná. Nyní je tato nová implementace zobrazována v procesu objednávky již 2 měsíce. 


Průměrně si placený tarif zvolí 9 \% zákazníků, kteří dokončí objednávku. Z toho nejdražší tarif si zvolí 1,5 \% zákazníků. Z veřejně dostupných dat je možné zjistit, že Kiwi.com mělo v roce 2018 obrat 30 miliard Kč. Průměrný měsíční obrat tedy činil 2,5 miliardy korun. V případě, že si 8,5 \% zákazníků objednalo tarif Standard a 1,5 \% vybralo nejdražší tarif Flexi, pak tržby z prodeje tarifů tvoří 33 milionů korun měsíčně. Nicméně je důležité uvést, že s prodejem tarifů plynou i náklady spojené s poskytováním náhrad a nadstandardních služeb. Pokud bychom tedy odhadly 70\% náklady, pak by čistý zisk tvořil 9 900 000 Kč. Pokud bychom náklady odhadly na 50 \%, pak by čistý zisk tvořil 16 500 000 Kč. Tímto velmi nepřesným způsobem odhadům nelze dosáhnout objektivního závěru. Nicméně lze ale s jistotou tvrdit, že společnost prodejem balíčků tarifů není ve ztrátě. Navíc implementace umožnila snadnou změnu koeficientů a lze tedy velmi snadno testovat a manipulovat s finanční stránkou nabídky tarifů. Základním údajem vzhledem k situaci je tedy procento zákazníků, které si vybere placený tarif. 

Průměrně si placený tarif zvolí 9 \% zákazníků, kteří dokončí objednávku. Z toho nejdražší tarif si zvolí 1,5 \% zákazníků. Z instituce BTS je známo, že průměrná cena letenky je v přepočtu 9 100 Kč. Při úvaze průměrné objednávky se dvěma pasažéry a tedy celkovou hodnotou objednávky 9 000 Kč, kde by Kiwi.com prodalo již odůvodněných 750 000 objednávek měsíčně, je celková cena objednávek 6 750 000 000. V případě, že si 8,5 \% zákazníků objednalo tarif Standard a 1,5 \% vybralo nejdražší tarif Flexi, pak tržby z prodeje tarifů tvoří 8 910 000 Kč měsíčně.  

Průměrná volba placeného doplňkového balíčku z dokončených objednávek je 4 \%. Cena balíčků je ovšem fixní a není závislá na hodnotě celkové objednávky. K určení tržby by bylo třeba znát celkový přesný počet objednávek. Opět narážíme na problém, že Kiwi.com si nepřeje tyto informace publikovat. Je nutné tedy použít znovu odhad počtu objednávek z veřejně dostupných informací. Z veřejně dostupných dat je možné zjistit, že Kiwi.com prodá 35 tisíc sedadel denně. Při průměrně 1,4 pasažérech v objednávce lze odhadnout 750 000 tisíc objednávek měsíčně. Na základě těchto údajů již můžeme odhadnout celkové tržby z prodeje balíčků, které mohou tvořit přibližně 972 000 Kč. 

Další velmi pozitivní informací je stabilní hodnota důležitého faktoru konverze objednávek. Při návrhu řešení se nabízely předpoklady, že celková konverze zákazníků, tedy počet zákazníků, kteří dokončí objednávku, bude nižší. Tento předpoklad se ani po 2 měsících uvedení nové funkcionality nepotvrdil. Konverze objednávek je na stejné úrovni, jako před implementací nové nabídky tarifů a balíčků. 

Překvapivé je ovšem, že narozdíl od úspěchu nové nabídky tarifů je prodej doplňkových balíčků poloviční ve srovnání s prodejem tarifů. Balíčky se přitom zobrazují ve velmi podobném vizuálním stylu jako tarify. Oproti původnímu řešení, kdy byl prodej placených balíčků pouze 1 \% je to ovšem zvýšení o 3 \%, což tvoří v rámci objemu objednávek významné číslo. Zvýšení prodeje může být odůvodněno prestižnějším místem v rámci procesu objednávky na samostatném krok objednávky, kde uživatel není rušen dalšími elementy a nabídkami. 

Nově implementovaná funkcionalita přinesla silně pozitivní ekonomické přínosy. Zisk, jako primární měřítko, byl jednoznačně novou nabídkou a funkcionalitou zvýšen. Je ovšem nemožné určit, zda dosažený přínos byl maximalizován. Existuje možnost, že lze zisk z nabídky tarifů a balíčků ještě zvýšit. Zvýšení zisku by mohlo být dosaženo například změnou designu, změnou pozice volby tarifu a balíčků z hlediska procesu objednávky, agresivněji propagovanou nabídkou důležitosti prémiových balíčků a jiných metod. Je ovšem nutné dbát opatrnosti i na další ukazatele, jako je počet dokončených objednávek, celková částka objednávek a další faktory, které se promítají na celkovou prosperitu firmy. 

Firma se tedy dále bude zabývat dosažením ještě vyššího zisku z nabídky balíčků a doplňkových služeb, protože nově implementovaná funkcionalita přinesla zisk a je možné, že míra dosaženého zisku může být pomocí drobných změn ještě vyšší. Budoucí implementace vylepšení by měla být testována a porovnávána oproti řešení popsaného v této práci. Testování je vhodné provádět pomocí AB testů, kde se rozdělí návštěvníci na skupiny. Jedné části je nabídnuta původní verze a druhé části vylepšená verze. Následně se analyzuje, který segment dosáhl lepších hodnot.