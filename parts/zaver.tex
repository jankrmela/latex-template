\kapitola{Závěr}
Cílem práce bylo vytvořit redakční systém, který by umožnil uživateli editovat jakoukoliv HTML šablonu z~pohodlí prohlížeče, bez znalostí programování. V~práci bylo postupováno systematicky od teoretické části, která vysvětlila problematické pojmy a~pomohla pochopení principu problému.

Analýzou současných řešení bylo zjištěno, že navrhovaná metodika je úplně nová a~podobnou myšlenku zatím nikdo neuskutečnil. V~současných řešení je vždy nutné upravit zdrojový kód HTML šablony pro konkrétní redakční systém. 

Byly stanoveny konkrétní požadavky, které by měla aplikace splňovat. Všechny byly vytvořené v~souladu s~očekáváním od cílové skupiny. Touto skupinou jsou především lidé bez hlubokých počítačových znalostí. Všechny požadavky tedy měly za cíl velké zjednodušení editace a~publikace webové stránky.

Na základě stanovených požadavků byla aplikace modelována za použití diagramů a~designových návrhů. Tento postup velmi zjednodušil následnou implementaci a díky tomuto návrhu nebylo nutné kód výrazně přepracovávat.

Po této analýze a~ujasnění si funkčnosti, bylo nutné zvolit vhodné nástroje. Bylo vybráno mnoho technologií a~nástrojů, které ve správné kombinaci umožnily dosáhnout optimálních výsledků.

Následně byla aplikace naimplementována. Implementace obsahuje nové řešení problémů a~nové algoritmy navržené pro konkrétní problém. Nebylo možné využít současných řešení, protože tuto problematiku ještě nikdo veřejně neřešil. Výsledná aplikace byla otestována automatickými testy i~manuálním hodnocením. 

Ukázalo se, že aplikace je vhodná na jednoduché weby jako jsou osobní prezentace, blog nebo webové stránky událostí a~akcí. Na těchto typech webu přináší velké zrychlení tvorby, zjednodušení editace a~teoreticky neomezené množství šablon, které lze použít. Uživatel se již nemusí vázat na omezený počet šablon, připravených pro konkrétní redakční systém.

Tato práce přinesla novou myšlenku při tvorbě webů, kdy se na proces vytváření webových stránek podívala přesně z~opačné strany. To přineslo velké zjednodušení pro koncového uživatele, ale také bylo vykoupeno složitou implementací. 