\kapitola{Závěr}

Cílem práce byla implementace nová nabídka tarifů a~doplňkových balíčků pro celosvětový portál Kiwi.com, která zprostředkovává prodej letenek tisícům zákazníkům denně. Nabídka tarifů v~původní verzi neexistovala. Její implementace měla kromě zvýšení zisku zároveň vyřešit několik problémů, které byly definovány firmou v~rámci původního řešení. Zisk se podařilo novou implementací tarifů opravdu zvýšit a~společnost Kiwi.com je s~přínosem velmi spokojená. Z~důvodu zákazu publikování čistého zisku a~dalších citlivých údajů, zde mohl být uveden pouze hrubý příjem z~prodeje tarifů, který činil téměř 9 milionů korun měsíčně. Nově implementované balíčky lze považovat také za úspěšné. Jejich prodej se zvýšil z~1 \% na 4 \% a~průměrně se prodají balíčky v~hodnotě 1 milion korun měsíčně. 

Řešení bylo zpracováno tak, aby vyřešilo další, předem definované problémy. Firma Kiwi.com měla potíže určit, v~jakých případech si zákazník zaslouží odškodnění v~případě dopravních komplikací, nečekaných událostí, chyb na straně zákazníka a~dalších problémech, při kterých zákazník požadoval finanční náhradu od společnosti. Tento problém byl vyřešen pomocí nabídky 3 tarifů, kde si uživatel sám v~rámci objednávky zvolí, jakou míru záruky od Kiwi.com požaduje. 

Novou nabídkou těchto tarifů byl zároveň poučen, jakou zodpovědnost může na společnost přenést, jaké služby má v~rámci tarifu a~kolik by ho stáli požadavky navíc. Tím byl vyřešen i~druhý definovaný problém - nízká informativní hodnota balíčků pro uživatele. Forma nabídky dle definovaných problémů měla zároveň zvýšit prodej doplňkových balíčků. Tento požadavek byl splněn vhodným umístěním nabídky na samostatný krok, kde uživatel měl na výběr mezi třemi tarify a~následně třemi balíčky a~nebyl rozptylován dalšími službami jako v~předchozím řešení. Uživatel tedy pro pokračování v~procesu objednávky musel jednoznačně zvolit o~který balíček má zájem, což jednoznačně zvýšilo prodej placených balíčků.

Podrobná analýza a~návrh řešení umožnil vyřešit i~poslední definovaný problém, kterým byla nutnost manuálně zpracovávat reklamační požadavky. Nyní je možné velkou část požadavků zautomatizovat a~tedy ušetřit náklady na poskytování zákaznické podpory. 

Není možné nově implementované řešení porovnat s~konkurencí, neboť analýza konkurence ukázala, že podobné řešení na trhu ještě neexistuje, což byl poměrně očekávaný stav, protože Kiwi.com je obrovská společnost s~miliardovým obratem a~prosazuje mnoho inovací v~oblasti prodeje letenek. 

V~technické rovině proběhla implementace pomocí nejnovějších nástrojů a~moderních principů. Integrace nového řešení bylo nutné vhodně zařadit do stávajícího repozitáře a~napojit komunikaci na již užívané rozhraní. Zároveň bylo potřeba vytvořit nové rozhraní pro získání a~zpracování nových dat k~nabídce tarifů. K~nové implementaci byly také naprogramovány automatické jednotkové i~end-to-end testy. 

Publikování nové funkcionality proběhlo opatrně pomocí AB testu, kde nové funkcionalitě bylo přiřazeno 50 \% návštěvnost. Z~dat bylo zřejmé, že nová nabídka tarifů a~balíčků je funkční a~přináší firmě značný zisk. Nebylo již tedy nutné rozdělovat návštěvnost a~AB test se vypnul v~prospěch nové funkcionality. 

Firma Kiwi.com je velmi spokojena s~dosaženými výsledky po finanční stránce, protože nové řešení přineslo podstatné zvýšení zisku. Ale jak již bylo detailně popsáno v~předchozích kapitolách, nelze určit, zda implementované řešení maximalizuje zisk a~další pozitivní efekty spojené s~objednávkovým procesem. Je možné, že lze dosáhnout ještě vyššího zisku nebo že lze nabízet služby ještě v~přehlednější a~příjemnější formě. Je tedy vhodné, aby firma Kiwi.com dále testovala různé varianty nabídky navržené v~kapitole Diskuze. Tato implementace nepochybně dokázala, že uživatelé mají o~tuto formu nabídky doplňkových služeb a~tarifů zájem. Zároveň tato funkcionalita přinesla firmě velmi významné zvýšení zisku. Pro společnost Kiwi.com je zde příležitost tuto nabídku ještě více monetizovat.
