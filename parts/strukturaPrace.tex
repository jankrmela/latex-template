\kapitola{Struktura práce}
Před zahájením jakéhokoliv dalšího postupu musí být objasněna tématika v teoretické rovině. Jelikož práce se zabývá implementací pro frontendovou část, tak bude předmětem teoretické částí architektura frontendu. Oblast frontendu je velmi rozsáhlá a v poslední době zažívá velmi rychlý rozvoj. Cílem zkoumání tedy je i zachycení trendů, aktuálních nástrojů a technik, které se v současné době používají. Jsou rozebrány i jednotlivé typy architektur, kde největší prostor je věnován právě typu architektury, který využívá i báze Kiwi.com, do které je nutné integrovat nové řešení.

Kapitola návrh a implementace řešení je rozdělena do více podkapitol, neboť se jedná o stěžejní zaměření této práce. Uvedení kapitoly je provedeno zpracováním vize firmy, kde na základě vhodného diagramu BSC je provedena analýza firmy ze čtyř různých perspektiv, na které je brán zřetel v dalších částech práce. Tato analýza je rovněž podpořena strategickou mapou k BSC diagramu.

Nezbytnou součástí je také podrobná definice definice problému, ve které jsou jednotlivé problémy detailně popsány včetně dopadů a příčin problémů. 

Analýza současného řešení popisuje průběh procesu, který začíná nalezení konkrétního letu po zaplacení objednávky. V analýze jsou definovány jednotlivé procesy, vstupní a výstupní podmínky, vnitřní a vnější události procesu včetně podnikových pravidel. Celková činnost procesu je následně znázorněna pomocí BPNM diagramu. 

Další podkapitolou je analýza konkurenčních řešení, kde jsou vybrány konkurenční firmy působící na stejném trhu, nebo nabízíjecí podobné služby ve stejně velkém rozsahu jako firma Kiwi.com. Analýza konkurenčních řešení obsahuje také důkladnou analýzu objednávkového procesu letenek, analýzu služeb, které nabízejí včetně rozložení jednotlivých elementů. Sumarizace konkurence pak je provedena v souvislosti s Kiwi.com formou SWOT analýzy.

V detailní analýze požadavků je vytvořen model návrh modelu, analýza požadavků a případy užití, proces tvorba objednávky a proces volby balíčků, který je opět znázorněn BPNM diagramem. Analýzu ještě doplňuje doménový a drátěný model s diagramem nasazení.

Implementace požadavků obsahuje specifikaci rozhraní, na které je nutné implementovat novou nabídku tarifů a balíčků. Jsou popsány jednotlivé atributy, které slouží pro komunikaci mezi backendovou a frontendovou částí. Popis životního cyklu aplikace zobrazuje přesně jak jednotlivé implementované části spolu souvisejí a komunikují. Pro přehlednost je také zobrazena stromová struktura důležitých nově implementovaných komponent. Součástí je také popis použitých knihoven třetích stran, struktura a proces testování pomocí automatických jednotkových a end-to-end testů. 

Vyhodnocení úspěšnosti řešení z ekonomického pohledu je uvedeno v kapitole ekonomické přínosy. Celkové zhodnocení a návrhy na budoucí vylepšení je uvedeno v kapitole diskuze. 
