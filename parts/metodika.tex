\kapitola{Metodika práce}
Po doplnění znalostí v teoretické rovině je nutné stanovit správný postup pro vytvoření úspěšného řešení. Prvním krokem bude analýza firemních požadavků s celkovým ohledem na budoucí vize firmy a její cíl s ohledem na všechny aspekty, které může řešení problému ovlivnit. K nalezení vhodného nástroje pro znázornění těchto požadavků, je zapotřebí zvolit správnou metodu z oblasti managementu. K doplnění znalostí o firemních cílech a plánech jsou vhodné materiály, které firma dává k dispozici zaměstnancům. 

Definováním konkrétního problému se umožní vytvořit relevantní analýzy současného řešení oblastí, na které je nutné se zaměřit. Společnost má velmi dobře definované jednotlivé aspekty produktu, funkcionalit a jak by měli jednotlivé části produktu spolu fungovat. Požadavkem je tyto dokumenty nastudovat a na jejich základě vytvořit zmíněnou analýzu současného řešení. 

Pro zjištění situace na trhu je nutné vytvořit detailní analýzu konkurence a zejména rozhodnout které subjekty lze považovat za konkurenci. Tyto subjekty následně důkladně prověřit v problematických oblastech. Situaci usnadňuje fakt, že se práce pohybuje v online prostředí a je snadné ověřit funkcionalitu přes webové aplikace daných společností. Výsledky analýzy, znázorněné například SWOT analýzou, umožní následný návrh řešení.

Návrh řešení, který bude sloužit jako základ pro následnou implementaci, je nutné znázornit pomocí vhodných nástrojů používaných pro návrh systému. Tato analytická část musí být v souladu se správnou metodologií při modelování procesů dané problematiky.

Nedílnou součástí návrhu řešení je i designový návrh, které vychází z návrhu drátěného modelu. Jelikož navrhované řešení nebude tvořit samostatnou jednotku, ale bude tvořit pouze modul, který navazuje na celý existují produkt, musí být vizuálně sjednocen dle grafických definic produktu. To jsou firemní předpisy, které detailně popisují pravidla, podle kterých se navrhují nové prvky. Designový návrh musí být zpracovaný ve firemně uznávaném nástroji Figma. Studium, jak ovládat předepsaný nástroj, je možné provést z online dokumentace na stránkách nástroje. 

Následná implementace do repozitáře firmy musí být provedena dle dobře dokumentovaných firemních postupů, které jsou dostupné na firemním úložišti. Studium těchto procesů, postupů a zásad o psaní kódu, je prerekvizitou pro zahájení implementace. 

Podmínkou pro implementaci funkcionality do repozitáře Kiwi.com je použití nejnovějších postupů, které vedou k optimálnímu, bezpečnému a čitelnému kódu. Tyto znalosti je možné získat studiem na odborných stránkách používaných nástrojů a každoročních konferencí, zabývající se webovým vývojem moderních aplikací. Je umožněno a silně doporučeno veškerou problematiku týkající se implementační částí konzultovat se zkušenějšími firemními programátory, kteří mají větší zkušenost s daným repozitářem.
