\kapitola{Metodika práce}
Tato kapitola uvádí postup při řešení problému. Stručně jsou popsány jednotlivé postupy a~důvody zvolení těchto postupů. Řešení problému lze rozdělit do tří hlavních skupin, jak již bylo naznačeno ve struktuře práce.

Před zahájením dalších kroků vedoucích k~řešení problému, byl proveden sběr informací z~literárních zdrojů, které pomohli pochopit následující důležité souvislosti. Nejdůležitější objekt této práce je redakční systém, proto byla velká část studia věnována právě redakčním systémům, jejich funkcím, dostupností a~porovnáním jednotlivých vlastností redakčních systému.


Při studiu byla nutná velmi podrobná znalost o~HTML, sémantice, struktuře i~historii. V~návaznosti na toto studium bylo prozkoumáno i~velké množství HTML šablon, které byly staženy zdarma z~internetu.

Po tomto studiu byla vypracována teoretická část, kde důležité poznatky, vázající se k~této problematice, byly na základě literárních zdrojů shrnuty do jednotné podoby. Takto zpracovaná teoretická část má za cíl pochopit a~dát do souvislostí informace nezbytné pro správné proniknutí do dalších částí této práce. Byly definovány principy HTML jazyka, konverze HTML jazyka a~srovnána současné řešení, které jsou dostupné online.


Následně byl detailně zpracován a~analyzován problém. S~tím je vázáno i~konkrétní definování požadavků. Při definování požadavků je velmi důležité věnovat zvýšenou pozornost, aby se jednotlivé požadavky shodovaly s~definovanými cíli. Také je potřeba dodržet hlavní zásady jako jsou měřitelnost, proveditelnost a~testovatelnost. Definice požadavků vychází primárně z~analýzy problémů a~snaží se jednotlivé problémy vyřešit. V~tomto případě se snaží najít mezeru na trhu redakčních systémů. Zjišťování funkcí, vlastností a~ceníku bylo prováděno na základě internetových stránek daných služeb.


V~reakci na definici požadavků byl vytvořen návrh řešení, který se dělí na několik pohledů. Výsledná aplikace a~její funkčnost byla shrnuta do neformální a~formální specifikace, která se dále dělí na funkční a~nefunkční požadavky. V~nefunkčních požadavcích byl na základě předchozí analýze stanovena funkčnost, kterou by mělo výsledné řešení vlastnit a~ve funkčních požadavcích byly stanoveny minimální požadavky na hardware. Konfigurace byla stanovena na základě aktuálních statistik náročnosti středně velkých webových aplikací.


Návrh modelu obsahoval propracovanou analýzu všech podstatných diagramů v~rámci této aplikace, pomocí kterých byla vizualizována funkčnost a~návaznost vzhledem k~jednotlivým částem aplikace. Při návrhu se vycházelo z~uvedených požadavků.


Design aplikace byl navržen dle současných trendů a~nových metod. Jako první byly vytvořeny drátěné modely, které určí rozložení jednotlivých prvků na stránce. Při tvorbě rozhraní byly brány ohledy na omezené počítačové schopnosti cílové skupiny, proto bylo vše navrhnuto maximálně jednoduše a~intuitivně.


Před zahájením implementační části se zjišťovali informace na výběr vhodných nástrojů a~metod. Informace byly získány především z~diskuzních fór odborníků a~odborných článků na internetu. Na základě tohoto informačního přehledu byly zvoleny technologie nejvhodnější pro implementaci aplikace. Zásadní části implementace byly shrnuty v~rámci podkapitoly řešení specifických problému. Jedná se převážně o~ukázky a~popis kódu, který je velmi důležitý k~vyřešení daných problémů, nebo se jedná o~úplně unikátní řešení, které ukazuje nový pohled na problém.


Při testování bylo zajištěno, aby veškeré testy proběhly v~souladu s~poučkami, dostupných v~literárních zdrojích. Typy testů byly zvoleny na základě rozsahu aplikace a~použitých nástrojů. Výsledky testů byly porovnány s~odhadovanými hodnotami. Implementace a~provedení testů velmi zjednoduší případné rozšíření aplikace.