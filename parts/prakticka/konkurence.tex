\sekce{Analýza současného řešení}

Obsahem této kapitoly bude analýza konkurenčních řešení se zaměřením na problémy, které byly shrnuty v analýze současného řešení. Dále bude zaměření na porovnání výsledku z již zjištěných podmínek, činností, pravidel a událostí ve stávajícím procesu. Předmětem strategie je specializace na sběr a analýzu dat z procesů a nabídky konkurenčních společností. Je to základní taktika pro zjištění, jak konkurence řeší podobné problémy, jak působí na zákazníky a jakou jsou hrozbou pro danou společnost.

Základním krokem je identifikace konkurence. Konkurence se identifikuje na základě podobnosti v rámci nabízeném produktu, uspokojování podobných potřeb a působením ve stejném odvětví. Přítomnost více konkurentů na jednom trhu může vést ke snižování cen služeb a zboží, protože se jednotliví konkurenti snaží o získání co největším podílu na trhu. Konkurenti jsou také více konkurence schopnější, pokud dokáží snižovat náklady a optimalizovat produkci ve srovnání se svými konkurenty na daném trhu. 

\podsekce{Konkurenční společnosti pro společnost Kiwi.com}
\paragraph{Definice konkurentů z pohledu Kiwi.com} \mbox{}

\noindent Společnost Kiwi.com se prezentuje jako internetová cestovní agentura s průlomovou technologií nazvanou Virtuální propojování. Virtuální propojování spočívá ve vyhledávání cesty, která není závislá na jednom dopravci. Kiwi.com dokáže kombinovat cestu pomocí stovek leteckých i pozemních dopravců. 

Kiwi.com působí na celosvětovém trhu, proto i analýza konkurence bude ze světového hlediska více relevantní, než jen pro Českou republiku. 

Mezi nejvýznamnější konkurenty patří Google flights a Scyscanner. Oba jsou definováni jako meta vyhledávači (metasearch). Meta vyhledávači agregují data od téměř všech virtuálních cestovních kanceláří. Nad takto agregovanými daty provádí analýzy a následně může svým uživatelům nabídnout nejvhodnější let pro jimi zvolenou destinaci. Tyto výsledky lze navíc optimálně řadit, například podle nejnižší ceny. Značně tak usnadňují hledání optimalní letenky. V případě, že by tyto vyhledávače nebyly k dispozici, uživatel by musel procházet všechny virtuální cestovní kanceláře manuálně a kombinovat různé možnosti letů. Manuálně by tedy bylo obrovsky časově náročné dosáhnout podobných výsledků jako s metasearch vyhledávači.

Analýza konkurence bude provedena pomocí matice, kde bude provedeno zobrazení konkurenčních firem a jejich služeb, které nabízejí v rámci procesu objednávky cesty. V případě, že společnost nabízí doplňkové služby bude znázorněno, jakým způsobem prezentují a nabízejí tyto doplňkové služby.

\podsekce{Základní přehled konkurentů}
V tabulce č. \ref{table:kiwiCompetit} je stručný přehled procesů a podmínek výběru přímých i nepřímých konkurenčních firem. Analýza se zaměřuje na základní fakta a parametry, které se přímo týkají uživatelů, používající dané služby. Rozbor byl obohacen o ukazatele, které zobrazují povědomí a spokojenost s uvedenými společnostmi a jejich aplikacemi. Výsledek hodnocení aplikací z App Store a Google play obvykle značí, pocity spojené s používáním aplikace. Uživatelé do tohoto hodnocení také často vyjadřují názor na společnost a služby. 

Dalším parametrem hodnocení je výsledek ze služby Trustpilot. Trustpilot je celosvětová služba, kde běžní zákazníci zanechávají recenze. Měsíčně se zde uloží přes jeden milion recenzí. 

\begin{table}[H]
    \includegraphics[width=\linewidth]{2020-04-27-20-18-58.png}
    \caption{Přehled konkurentů pro společnost Kiwi.com}
    \label{table:kiwiCompetit}
\end{table}

\podsekce{Rozdělení dle možností dokoupení doplňkových služeb}

Tabulka č. \ref{table:packagesCompetition} s konkurenčními firmami a jejich nabídky doplňkových služeb se zaměřuje na nabízené typy doplňkových služeb. Společnosti Google flights a Skyscanner jsou odlišné ve svém obchodním modelu, jak již bylo nastíněno v předchozích sekcích. Není tedy objektivní uvést, zda obsahují nebo neobsahují nákup konkrétních doplňkových služeb. Nicméně i přes to, že na svých stránkách při výběru letenky tuto možnost nenabízí, tak je zákazník přesměrován na stránky partnera, kde již tyto doplňkové služby uživatel může dokoupit. Nabídka se tedy liší v případě těchto dvou společností v rámci různých letů v závislosti na partnerech provozující daný let.

Dále je již konkurence ve své nabídce služeb poměrně vyrovnaná. Všechny společnosti nabízí cestovní pojištění i výběr sedadel. Na porovnání způsobu nabízení služeb se zaměříme v následujících odstavcích. Společnost Kiwi.com jako jediná z uvedených konkurentů nabízí online odbavení. Doplňkové balíčky má v nabídce GoToGate i Kiwi.com. Formou prezentace nabídky balíčků i obsahem se od sebe příliš neliší. I tato část bude podrobena rozboru v jednom z dalších odstavcích analýzy konkurence.

\begin{table}[H]
    \includegraphics[width=\linewidth]{2020-04-27-20-43-12.png}
    \caption{Nabízené služby konkurenty}
    \label{table:packagesCompetition}
\end{table}

\paragraph{Ukázka nabídky pojištění společností GoToGate} \mbox{}

\noindent V rámci posledního kroku před platbou je uživatel dotázán, zda má zájem o pojištění storna. V případě, že si uživatel tuto službu vybere, má nárok na zrušení letenky a náklady s tím spojené zákazníkovi uhradí společnost GoToGate. Nabídka může působit, že zákazník má právo na storno v jakémkoliv případě. Překvapení ale může nastat po zobrazení podmínek tohoto pojištění. Vztahuje se pouze na lékařsky potvrzené zdravotní komplikace jednoho z pasažérů. 

\obrazek\vlozobrbox{2020-04-27-21-00-49.png}{0.9\textwidth}{!}\endobrl{GoToGate -- nabídka pojištění}{gotogate}

Z UX hlediska je zákazník doslova nucen zvolit jednu z variant, protože další krok je blokován, dokud neprovede volbu. Ani jedna z možností není předvyplněna. Toto chování webové aplikace může zapůsobit tak, že se uživatel opravdu musí rozhodnout o potřebnosti tohoto pojištění. Je velmi pravděpodobné, že je tímto chováním zvýšená konverze výběru placené verze. Stránka vykreslí následující chybu v případě, že uživatel chce pokračovat bez provedení volby.

\obrazek\vlozobrbox{2020-04-27-21-01-27.png}{0.9\textwidth}{!}\endobrl{GoToGate -- agresivní chybové hlášky}{gotogate2}

\paragraph{Ukázka nabídky flexibilní letenky společností GoToGate} \mbox{}

\noindent Další druh pojištění nabízí prvek nazvaný flexibilní letenka. Jedná se o placenou službu, která zajistí, že v případě potřeby je možné před odletem změnit zakoupený termín za jiný termín. Tato služba se nachází na prvním kroku objednávky.  

\obrazek\vlozobrbox{2020-04-27-21-02-36.png}{0.9\textwidth}{!}\endobrl{GoToGate -- flexibilní letenka}{gotogate3}

\paragraph{Ukázka nabídky doplňkových balíčků společností GoToGate} \mbox{}

\noindent Doplňkové služby jsou nabízeny velmi podobným způsobem, jako je nabízí společnost Kiwi.com. V nabídce se nachází 3 balíčky. První balíček je zdarma a neobsahuje žádné služby společnosti zdarma. V případě, že by zákazník potřeboval pomoc nebo podporu ze strany GoToGate by musel zaplatit administrativní poplatek 490 Kč. Druhý balíček obsahuje kromě rychlé refundace všechny služby podpory v ceně. Poslední a zároveň nejdražší balíček obsahuje všechny služby podpory včetně rychlé refundace. 

Z vizuálního hlediska je zvýrazněn střední varianta balíčků, která má za cíl zvýšit pozornost uživatele a následně její výběr. Tento fakt je podpořen doplňkovým textem pod balíčkem, který oznamuje, že tento balíček zvolila více než polovina zákazníků. Volba mezi balíčky se nachází na kroku před platbou. Volba je povinná a bez jejího učinění není možné pokračovat.

\obrazek\vlozobrbox{2020-04-27-21-03-48.png}{0.9\textwidth}{!}\endobrl{GoToGate -- nabídka balíčků}{gotogate4}

V případě, že uživatel vybere nejlevnější balíček okamžitě se zobrazí vyskakovací okno, které překryje obsah stránky. Uživatel je informován o výhodách Premium balíčku a které služby v případě zakoupení obdrží. Je zvýrazněno tlačítko pro zavření vyskakovacího okna, které zároveň přidá uživateli prémiový balíček do košíku. 


\paragraph{Ukázka nabídky pojištění společností Expedia} \mbox{}

\noindent Společnost Expedia nenabízí doplňkové služby, ale uživatel má možnost dokoupit pojištění letu. Pojištění letu obsahuje možnosti vrácení peněz v několika případech, které jsou detailně popsány přímo zobrazeny nad volbou pojištění. Uživatel musí zvolit zda má zájem, nebo nemá zájem. Bez této volby není možné rezervovat let. Volba, která obsahuje pojištění a tedy i zvyšuje cenu letu je zvýrazněna zelenou barvou, která na uživatele může pozitivně působit.  

\obrazek\vlozobrbox{2020-04-27-21-05-05.png}{0.9\textwidth}{!}\endobrl{Expedia -- nabídka pojištění}{expedia}

Pokud uživatel nevybere ani jednu z možností a pokusí se dokončit rezervaci stisknutím potvrzovacího tlačítka, tak stránka zareaguje zvýrazněním tohoto elementu na stránce červenou barvou. Tato signalizace nemusí působit dobře na uživatele, protože červená barva vyvolává v lidech pocit nebezpečí.

\obrazek\vlozobrbox{2020-04-27-21-05-50.png}{0.9\textwidth}{!}\endobrl{Expedia -- znázornění blokování dalších akcí}{expedia2}

\podsekce{Shrnutí analýzy konkurence}

Byly vybrány konkurenční společnosti pro firmu Kiwi.com a popsány a porovnány jejich nabízené služby, které mají přímou souvislost s návrhem řešení v rámci této práce. Při porovnávání se uvažovali atributy, podmínky a procesy, které byly předmětem zkoumání i u současného řešení. 

Následuje shrnutí jednotlivých firem slovním popisem, kde jsou vyzdvihnuty jejich pozitivní a negativní stránky. Toto shrnutí pomáhá stanovit optimální návrh nového řešení s ohledem na situaci v konkurenčním prostředí.

\paragraph{Google flights} \mbox{}

\noindent Kombinují skvělé funkce, které zanechávají skvělý uživatelský dojem. Webová aplikace je přehledná a velmi svižná. Vzhledem k tomu, že se zaměřují na mírně odlišný obchodní model, mají značnou výhodu v oblasti personalizace. Google, jako firma, která stojí za touto službou, má pro skvělou personalizaci velké množství dat, na základě kterých velmi dobře oslovit zákazníka s nabídkou na míru. 
Specializují se pouze na provizi z nákupu letenky u svých partnerských společností, kam přesměrovávají zákazníky pro dokončení objednávky. Jejich funkcionalita tedy hlavně spočívá ve vyhledávání letenek a agregování dat. Dokáží také přehledně zobrazit statistiky cen u konkrétních destinací vzhledem k časovému období. Nelze uvažovat nad definovanými problémy pro službu Kiwi.com, jelikož Google flights nenabízí balíčky žádným způsobem. Řešení pro Kiwi.com tedy bude v porovnání s Google flights unikátní.

\paragraph{Skyscanner} \mbox{}

\noindent Jeden z velmi populárních vyhledávačů na letenky s velmi podobným obchodním modelem jako Google flights. Narozdíl od Google flights mají pro zákazníky k dispozici i aplikace pro mobilní telefony s velmi dobrým hodnocením v rámci virtuálních obchodů na obou platformách. V aplikaci má uživatel v přehledném rozhraní vyhledat a filtrovat lety. Následnou rezervaci pak ale dokončí na webu partnera Scyscanneru, který zajišťuje spojení pro dané destinace. Na tuto rezervaci je přesměrován po tom, co má možnost zjistit potřebné informace ještě před opuštěním Scyscanneru. Skyscanner tedy nenabízí takový rozsah služeb jako Kiwi.com a definované problémy pro službu Kiwi.com nedokáže tento problém řešit.

\paragraph{Expedia} \mbox{}

\noindent Přímá konkurence Kiwi.com, která ovšem nabízí v rámci rezervace doplňkových služeb omezenější nabídku. S nadstandardních služeb nabízí pouze pojištění. Celý proces rezervace se odehrává pouze v rámci jednoho kroku. Uživatel má tedy mnoho elementů na jedné stránce a je pro něj těžké se soustředit na jednotlivé nabídky. Dokončení bookingu tedy může být rychlejší, než například na konkurenční službe GoToGate, protože se uživatel nemusí proklikávat množstvím doplňků. Doplněk pojištění cesty má však povinnou volbu, tudíž uživatel musí aktivně projevit nesouhlas, aby mohl dokončit objednávku. Jelikož Expedia balíčky nenabízí, tak ani nelze srovnávat definované problémy a budoucí řešení s Kiwi.com. 

\paragraph{GoToGate} \mbox{}

\noindent Služba, která nabízí velmi podobnou doplňkovou podporu, jako Kiwi.com. Objednávkový proces je rozdělen také do několika kroků, kde u každé služby musí uživatel vybírat, zda o ni má zájem, nebo nemá. S UX hlediska stránka obsahuje několik problému. V rámci provádění objednávky a procházení jednotlivých kroků není možné zpětně editovat předchozí kroky. Rozhraní neumožňuje vrátit se zpět. Při použití nativního tlačítka zpět v prohlížeči je uživatel přesměrován zpět na vyhledávání letů a celá rezervace je tímto ztracena. GoToGate nicméně jako jediná nabízí balíčky, nicméně v dosti omezeném rozsahu. V celkovém kontextu nejsou příliš zasazeny do jednotnosti produktu. Nabídku tarifů služba GoToGate neobsahuje vůbec a navrhované řešení bude i v porovnání s touto službou unikátní.

\podsekce{Zhodnocení předpokladů pro řešení Kiwi.com}

\begin{table}[H]
    \includegraphics[width=\linewidth]{2020-04-27-21-14-33.png}
    \caption{SWOT -- Analýza předpokladů pro Kiwi.com}
    \label{table:swot}
\end{table}
