\sekce{Definice problému}

V předchozí kapitole byly pomocí Balanced scorecard diagramu stanoveny vize, na základě kterých vyplývají konkrétní problémy, které mají být řešeny. Bylo definováno několik problémů se kterými se firma musí potýkat v rámci procesů zpracování objednávek a následném zákaznickém servisu. Problémy mají přímý dopad na zákazníka a jeho uživatelský požitek při používání portálu Kiwi.com.

\paragraph{Nejednotné měřítko při poskytování náhrad zákazníkům} \mbox{}

\noindent Procesní perspektiva BSC diagramu stanovila problém v oblasti rozhodování o náhradách zákazníkům. V případě, že u zákazníka nastává problém s letem, například nestihl letadlo, onemocněl, nebo byl let zrušen, je se zákazníkem zahájeno řízení pracovníkem zákaznické podpory. Zákaznická podpora pak podle různých okolností rozhoduje na jaké odškodnění má zákazník nárok. Nejsou tedy definovány jasné pravidla a v tomto procesu může nastat lidský omyl.

\paragraph{Nízká informativní hodnota balíčků pro uživatele} \mbox{}

\noindent Ze zákaznické perspektivy BSC diagramu vyplynul problém o seznamování zákazníka s nabízenými balíčky. Dle firmy a jejího výzkumu provedeném na uživatelích současná nabídka balíčků podpory působí zmateně. Není jim jasné, proč by si nadstandardní služby měli zakoupit. Rovněž jim není zřejmé, co všechno pokrývá záruka Kiwi.com v případě nutnosti změnit datum odletu, nebo úplně zrušit let.

\paragraph{Nízký prodej balíčků} \mbox{}

\noindent Ve finanční perspektivě BSC diagramu bylo na základě vize firmy plánováno vhodné snižování cen letenek. Cenu letenek by bylo možné snížit, pokud by se zvýšila prodejnost doplňkových balíčků  V návaznosti na předchozí problém s nízkou informativní hodnotou balíčků souvisí i celkový nízký počet zákazníků, kteří si vybírají nadstandardní služby. Dle firemních dat se jedná o jedno procento všech objednávek. Firma se domnívá, že toto číslo je nízké a je možné ho zvýšit.

\paragraph{Manuální řešení problémů zákazníků} \mbox{}

\noindent Procesní perspektivou BSC diagramu bylo plánováno efektivní zpracování poskytování služeb zákazníkům. Současný stav ale nenaplňuje vizi firmy. Kompletní proces poskytování náhrad je závislý na lidské pracovní síle a není automatizovaný. Je nutné rozhodovat ručně, který zákazník má právo na poskytnutí finanční náhrady za svůj let, nebo poskytnutí služeb, které mu pomohou let rezervovat na jiný termín. Tento proces by bylo vhodné automatizovat, pokud dojde k vyřešení první problému s jednotným měřítkem při poskytování náhrad zákazníkům.


