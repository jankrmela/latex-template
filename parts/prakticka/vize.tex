\sekce{Vize firmy}

\podsekce{Diagram BSC}
Balanced scorecard je vizualizace určené k měření efektivnosti aktivit vzhledem ke strategickým plánům společnosti. Používá se ke strategickému plánování, tak aby se zajistil soulad s celkovou strategií firemní vize a celkovými cíli firmy. Tato vizualizace byla vytvořena, aby pomohla hodnotit jednotlivé činnosti se širším pohledem, než s pouhým pohledem na ziskové měřítko. 

Diagram na obrázku \ref{bsc} bere v potaz čtyři hlavní perspektivy, na které je vhodné se zaměřit. První perspektiva je již zmíněná finanční perspektiva. Zaměřuje se na výnosnost investice, růst, náklady, zisk a podobně. 

Další je Procesní perspektiva. Tato perspektiva se zaměřuje na interní procesy firmy a jak se s těmito procesy bude strategie protínat a ovlivňovat. Ideálním stavem, který firma chce dosáhnout je zjednodušení a zrychlení procesů, neboť jsou přímo svázány s finančními prostředky. Zjednodušení procesů může zrychlit školení nových zaměstnanců, zmenšit chybovost, nebo zlepšit celkový výkon pracovníků.

Zákaznická perspektiva je třetí perspektivou. Zobrazuje změny vnímané zákazníkem a jak tyto změny zákazníka ovlivní. V zájmu firmy je vyvolat pozitivní působení, nebo alespoň neutrální, pokud dané změny pozitivně ovlivňují ostatní perspektivy. Popisovány jsou procesy z pohledu zákazníka. Berou se do úvahy jeho reakce, recenze a podněty. 

Perspektiva vývoje a růstu zkoumá působení změn v rámci celkového vývoje firmy a interních zaměstnanců. Předmětem rozboru může být i jak rychle je schopný tým adaptovat nastávající změny a jak na takové změny reagují. Inovační technologie působí zpravidla pozitivně v rámci vzdělávání a kariérního růstu. 

\begin{figure}[H]
    \centering
    \vlozobrbox{2020-04-27-18-10-22.png}{0.9\textwidth}{!}
    \caption{Diagram BSC}
    \label{bsc}
\end{figure}

\paragraph{Finanční perspektiva} \mbox{}

\begin{itemize}
    \item Zvýšení zisku z přímého prodeje balíčků 
    
    V současnosti jsou balíčky nevhodně interpretovány. Jejich prodejnost je nízká zejména z důvodu, že uživatelé nerozumí, co vlastně firma nabízí a proč by si doplňkové balíčky měli zakoupit. Dosažením cíle je očekáván větší zisk a cíl je tedy měřitelný.

    
    \item Snížení nadbytečných nákladů na zákaznickou podporu

    Zákaznická podpora je k dispozici všem zákazníkům, kteří si zakoupí let. Na základě jiných cílů uvedených dále je zřejmé, že zákazníci si nejsou jistí v jakých ohledech jim společnost může pomoc a jaké služby mají v ceně. Stanovení jasných sazeb v rámci balíčků je tento problém řešitelný.
    
    \item Možnosti snížení cen samotných letenek

    Cíl s ohledem na konkurenci, která se snaží snižovat základní ceny letenek. Společnost se následně snaží získat ušlý zisk na doplňkových službách, jako jsou zavazadla nebo prémiové služby. Zákazník ale v první chvíli procesu vybírá vhodnou společnost podle nejnižší základní ceny. Porovnávat celkovou cenu se všemi doplňky je už pro zákazníka náročnější. Snížení ceny samotné letenky je možné pomocí zavedení doplňkových balíčků. Služby jsou značně omezeny v rámci samotné letenky a cenu je tedy možno snížit.
\end{itemize}

\paragraph{Zákaznická perspektiva} \mbox{}

\begin{itemize}
    \item Seznámit zákazníka s nabízenými balíčky
    
    Zákazník je v rámci nabízených balíčků seznámen na jaké služby má v rámci ceny nárok a jaké služby je vhodné dokoupit s ohledem na jeho požadavky, typ cesty a podobně.
    
    \item Možnost přizpůsobit si služby na míru

    Každý zákazník má specifické požadavky. Vhodným rozdělení služeb do balíčků si může každý zákazník vybrat přesné části o které má zájem. V případě, že všechny nabízené služby byly součástí každé letenky, by se cena zvýšila, což by neuspokojilo skupinu zákazníků preferující co nejlevnější letenky.
    
    \item Obeznámení o podmínkách letu a náhradách

    Z předchozích zkušeností firmy se zjistilo, že zákazníci často nejsou seznámeni s podmínkami letu a v jakých případech mohou kontaktovat firmu, aby poskytla náhradu při případných problémech, které vznikly na straně aerolinky, nebo zákazníků samotných.

\end{itemize}

\paragraph{Perspektiva vývoje a růstu} \mbox{}

\begin{itemize}
    \item Zpřesnění výpočtu budoucí situace
    
    Jasně definovanými podmínkami náhrad zákazníkům, které si sami vybírají v rámci balíčků je možné přesněji odhadovat výdaje spojené s poskytováním budoucích náhrad v případech problémů. Analytické vyhodnocování v těchto případech nabývá přesnější odhady na základě známých dat z procesu nákupu.
    
    \item Zjednodušení složitosti produktu

    Přehledně zobrazenými definicemi v jakých případech je firma povinna poskytnout náhrady se snižuje proces zaučování do složitostí produktu. Nové zaměstnance je pak časové méně náročné školit. Zaměstnanci nejsou zahlcení negativními pocity a celková efektivnost je na vyšší úrovni.
    
    \item Optimalizace nabídky služeb pro specifické trhy

    V rámci specifických požadavků trhu je možné měnit nabídku podle požadavků, které více vyhovují danému trhu. Je možné vytvořit konkurenčně zajímavou nabídku. Situace na rozdílných trzích se mění i v čase. Je tedy vhodné mít možnost reagovat na tyto změny přizpůsobení nabídky.

\end{itemize}

\paragraph{Procesní perspektiva} \mbox{}

\begin{itemize}
    \item Jednoznačnost při rozhodování o náhradách zákazníkům
    
    Výrazné zjednodušení procesu by nastalo při jasně definovaných podmínkách náhrad zákazníkům. Takto zjednodušené procesy je možné řešit i automaticky bez zásahu, nebo s minimálním zásahem, lidského faktoru. 
    
    \item Snížení vytíženosti zákaznické podpory

    Zákaznická podpora je vytížena z velké části požadavky, které nejsou kompetentní z hlediska jejich pravomocí na poskytování pomoci či náhrad. Při zamezení nesprávných požadavků by se značně zvýšila kapacita a zároveň snížily celkové náklady.
    
    \item Efektivnost při zaučování nových zaměstnanců

    Nové zaměstnance je třeba zaučit do procesů, které jsou jim neznámé. S rostoucí složitostí roste i doba zaučení nového zaměstnance. S ohledem na vysokou fluktuaci zaměstnanců je důležité držet tuto dobu na co nejnižší úrovni.

\end{itemize}

\podsekce{Strategická mapa k BSC}
K diagramu balanced scorecard je vyhotovena strategická mapa, která znázorňuje jednotlivé perspektivy a vazby mezi nimi. V rámci této strategické mapy jsou zobrazeny konkrétní vazby všech prvků a způsob, jak se navzájem ovlivňují. Firma působí jako celek a součástí vize jsou i přesné cíle, které navazují na vizi. Jednotlivé cíle jsou mezi sebou propojené a ovlivňují firmu z různých perspektiv. Je nutné tedy nahlížet na tuto definici jako na celek.

\begin{figure}[H]
    \centering
    \vlozobrbox{2020-04-27-19-01-37.png}{0.9\textwidth}{!}
    \caption{Diagram strategické mapy k BSC}
    \label{mapaBSC}
\end{figure}