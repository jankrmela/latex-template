\kapitola{Použitá terminologie}
V~této kapitole jsou stručně popsány základní informace pro správné pochopení práce a~ujasnění použitých termínů. Při psaní práce je vycházeno převážně z~tištěné literatury, internetových zdrojů a~oficiální dokumentace k~jednotlivým nástrojům, které jsou v~průběhu procesu použity. Oficiální dokumentace občas není ke všem nástrojům k~dispozici. V~takovém případě je jediným řešením oficiální API dokumentace, kterou už každý kvalitní nástroj zveřejněnou má.

\sekce{Definice principů HTML jazyka}
HTML (Hypertext mark-up language) je sada daných značek, symbolů a~kódů, vložených do dokumentu, který slouží pro zobrazování webových stránek. Každý element (tag), definuje strukturu webu, nebo její části \cite{duckett}. Prohlížeče nezobrazují uživatelům tyto elementy, ale přímo prezentují obsah, zapsaný v~tomto jazyce. HTML elementy jsou většinou párové. Tato vlastnost je i~využita v~této práci, při rozpoznávání struktury elementů \cite{boehm}.

Jednotlivým elementům lze přiřazovat atributy. Atributy se zapisují přímo dovnitř elementů, hned za úvodní název elementu. Oddělují se mezerou. Jeden z~velmi důležitých atributů je unikátní identifikátor. Značí se pomocí klíčového slova id. Tento atribut hlavně využívá javascript a~CSS pro komunikaci s~daným elementem \cite{gauchat}. Velmi podobný je atribut třídy - class. Stejnou třídu lze přiřadit více elementům a~jeden element může mít přiřazeno více tříd.

\begin{htmlcode}
<div id="content" class="full-width">
    ...
</div>
\end{htmlcode}

Z~pohledu datové struktury a~teorie grafů jsou HTML elementy v~dokumentu uspořádány jako strom \cite{treew3}. Tomu je i~přizpůsoben algoritmus procházení a~analýzy HTML souborů, který je popsán v~této práci.
 

\obrazek
\vlozobrbox{html-strom.png}{0.9\textwidth}{!}
\endobrl{Stromová HTML struktura}{html-tree} 

\sekce{Konverze HTML jazyka}
Konverze probíhá modifikováním HTML souborů do PHP podoby, která lze editovat z~prohlížeče. Při konverzi je nutné dodržet několik zásad, aby výsledný soubor byl kompatibilní se všemi použitými nástroji zaručující požadovaný výsledek. 

S~ohledem na možnosti editoru v~prohlížeči musí být části stránky rozdělené do jednotlivých editovatelných bloků. O~správné rozpoznání editovatelných částí stránky a~její rozdělení do jednotlivých bloků se stará konvertor. V~konvertoru se obsah editovatelných bloků ukládá do databáze a~nahrazuje se správnou syntaxí, která v~případě potřeby, je schopna do stránky dosadit požadovaný obsah. Tyto bloky potřebují do svého nadřazeného elementu vložit identifikátor. Konkrétní identifikátor následně slouží pro uložení daného bloku k~původní instanci.   

Vzhledem ke stromové struktuře HTML je vhodné využít rekurzivní algoritmus pro analýzu celého dokumentu \cite{lorentz1994recursive}. Takový algoritmus může být při velké a~složité HTML struktuře časově náročný \cite{hylmar2009}.

\sekce{Porovnání konvertorů}
V~současné době není na internetu volně k~dispozici aplikace, která by byla schopná automaticky, bez zásahu uživatele převést HTML šablonu do systému, kde by byla možná editace pouze z~prohlížeče, bez znalosti programování.

Existují řešení, nabízející editaci z~prohlížeče, ale jen s~nutností předchozí úpravy HTML souborů. Uživatel bez znalostí programování není schopen zveřejnit a~upravovat staženou HTML šablonu pomocí současných řešení. V~této podkapitole jsou srovnány jednotlivé alternativy.

\paragraph{Surreal CMS} \mbox{} \\
Tento redakční systém umožňuje editaci vlastní webové šablony z~prohlížeče s~WYSIWYG editorem. Před editací je ale nutné všechny zdrojové soubory šablony upravit. Každému elementu, který uživatel chce upravovat, musí přidat speciální třídu \cite{surrealcmscite}. K~této aplikaci není potřeba vlastnit hosting. Vše běží na serveru poskytovatele. Cena tohoto řešení přijde uživatele na 30 dolarů měsíčně, při středně velkých webových stránkách. 
Aplikace se nachází na adrese http://www.surrealcms.com.

\obrazek
\vlozobrbox{surreal.png}{0.9\textwidth}{!}
\endobrl{Ukázka webu Surreal CMS}{surreal} 

\paragraph{CushyCMS} \mbox{} \\
Velmi podobná aplikace jako předchozí Surreal CMS. I~zde je nutná editace HTML souborů pomocí přidání atributů třídy. Měsíční poplatek je 28 dolarů.
Aplikace se nachází na adrese www.cushycms.com.

\obrazek
\vlozobrbox{cushy.png}{0.9\textwidth}{!}
\endobrl{Ukázka webu CushyCMS}{cushy}


\paragraph{Běžné redakční systémy} \mbox{} \\
Oblíbené redakční systémy jako Wordpress, Joomla, nebo Drupal nejsou vhodné k~porovnání. Tyto redakční systémy nesplňují hlavní požadavek a~smysl této práce - možnost použití HTML šablony v~neupravené podobě. Zmíněné redakční systémy podporují pouze speciálně upravené šablony pro konkrétní redakční systém. Takovou úpravu by běžný uživatel nezvládl. Jediná možnost je pak využít již vytvořené šablony pro tyto redakční systémy, kterých ale není mnoho. I~z~toho důvodu lze vidět spoustu stejných webových stránek pro různé firmy - jen s~jiným obsahem.


Dalším problémem pro běžné uživatele může být instalace tohoto redakčního systému na vlastní hosting. I~pro tyto činnosti méně zkušení uživatelé často kontaktují profesionální firmy, které jim pomohou.

V~této práci je redakční systém velmi důležitý pojem. Cílová skupina aplikace jsou lidé, kteří nemají hluboké počítačové znalosti a~právě redakční systém této skupině lidí velmi usnadňuje editování webových stránek. Redakční systém používá podobný princip úpravy webové stránky jako textové editory. 

\sekce{Návrh řešení webové aplikace}
Analýza webové aplikace nebo informačního systému lze provést pomocí několika metod. Velmi často se používají strukturované a~objektové analýzy. Strukturovaná analýza je starší než objektová a~používá se z~ní zejména funkční a~datové modelování \cite{larman2005applying}.


Poté, co se začalo programovat i~objektově orientovaným způsobem, se začala používat i~analýza, která pomohla osvětlit problém z~toho objektově orientovaného hlediska. Tento nový pohled představil i~nové metody pro analýzu \cite{buchczuml}. Sloučení funkční a~datové stránky aplikace je podstatou objektově orientovaného návrhu.


Pro kvalitní návrh je vhodné použít nástroje, které umožní správný a~komplexní pohled na daný problém. Tyto požadavky splňuje jazyk UML. Jazyk UML definuje jasná pravidla pro vytváření a~modelování diagramů. Různé diagramy přináší různé pohledy na problém. Jedině tak je možné dosáhnout optimálního návrhu. \cite{podeswa2010uml}

\sekce{Redakční systém}
Redakční systém je webová aplikace, které má rozhraní pro správu, konfiguraci a~editaci webové stránky. Její hlavní účel je, aby si každý uživatel mohl sám upravovat webovou stránku, bez nutnosti delegovat tuto práci programátorům \cite{verens2010cms}. I~programátoři preferují editaci webových stránek z~redakčního systému, protože často je taková editace komfortnější a~rychlejší, než editace HTML souborů \cite{cmsfordeveloper}.

\obrazek
\vlozobrbox{cms-compare.png}{0.9\textwidth}{!}
\endobrl{Oblíbenost redakčních systémů (Převzato z~websitesetup.org)}{cms-compare}

Nejpopulárnější redakční systém v~roce 2017 byl Wordpress \cite{wordpresstat}. I~přes technické nedokonalosti a~bezpečnostní díry jej využívalo, dle statistik Google Trends, 239 milionů webových stránek.


Následující obrázek popisuje obvyklý princip implementace šablony do redakčního systému. Tuto metodiku využívají všechny výše uvedené redakční systémy, včetně nejpopulárnějšího Wordpressu.

\obrazek
\vlozobrbox{my-process.png}{0.9\textwidth}{!}
\endobrl{Současný proces v~redakčních systémech}{myprocess}